%% LyX 2.1.4 created this file.  For more info, see http://www.lyx.org/.
%% Do not edit unless you really know what you are doing.
\documentclass[12pt,english]{article}
\usepackage{charter}
\renewcommand{\familydefault}{\rmdefault}
\usepackage[T1]{fontenc}
\usepackage[utf8]{luainputenc}
\usepackage{babel}
\usepackage{setspace}
\onehalfspacing
\usepackage[unicode=true,pdfusetitle,
 bookmarks=true,bookmarksnumbered=false,bookmarksopen=false,
 breaklinks=false,pdfborder={0 0 1},backref=false,colorlinks=false]
 {hyperref}

\makeatletter
%%%%%%%%%%%%%%%%%%%%%%%%%%%%%% User specified LaTeX commands.
\usepackage{babel}

\makeatother

\begin{document}

\section{Overall Description}


\subsection{Product Perspective}

The platform aims at replacing the traditional means of communication
between taxi drivers and customers, and it is therefore designed as
a stand-alone entity that is able to operate with a minimum set of
dependencies towards external systems. 

The external interfaces on which the system relies, in order to be
operational, are described in the following subsections.


\subsubsection{Software Interfaces \label{sub:Software-Dependencies}}
\begin{itemize}
\item Google Maps API: used in both the back-end and the taxi-side applications,
this free API is necessary to elaborate routes and display maps. The
system interacts with the API through standard GET requests.

\begin{itemize}
\item Mnemonic: elaborate routes and display maps
\item Source: https://developers.google.com/maps/
\end{itemize}
\item GPS: this well known system is required by the mobile applications
to get information about the users' position. The interaction happens
through the standard GPS protocol provided by the OS of the users'
devices.

\begin{itemize}
\item Mnemonic: get information about the user position
\item Source: devices' OSs' protocols
\end{itemize}
\item Red Hat Enterprise Linux: operating system run on the back-end server.

\begin{itemize}
\item Mnemonics: back-end OS
\item Source: https://access.redhat.com/documentation/en/red-hat-enterprise-linux/
\end{itemize}
\item Apache WebServer: software bundle run on the back-end, which is used
to allow interaction between the back-end system and the user-side
application. It is used to handle standard HTTP requests from the
customer-side web application and to provide CGI services to all the
user-side applications.

\begin{itemize}
\item Mnemonics: web-server
\item Source: https://httpd.apache.org/docs/2.4/
\end{itemize}
\item SQLite: DBMS software component run on the back-end, which is used
to store user data in a non-resource-intensive fashion.

\begin{itemize}
\item Mnemonics: DBMS
\item Source: https://www.sqlite.org/docs.html
\end{itemize}
\end{itemize}

\subsubsection{Device support}

Device support: the back-end system is designed to be run on usual
server hardware with a high fault-tolerance degree and supports hardware
scaling (such as replication); the mobile applications are designed
to be run on ARM based architectures; the user-side web application
does not have specific hardware requirements, since it relies on a
different level of abstraction (web browser) than its mobile counterpart
which is not to be taken into account in the design of the hardware
infrastructure.


\subsubsection{User interface}


\subsubsection{Memory constraints}

It is assumed that enough memory will be provided to the system in
order to scale in relation to the user traffic. More details on the
topic will be provided in the following sections. 


\subsection{Product functions}

The fully developed platform should implement the following high level
functions. 
\begin{itemize}
\item []Customer-side application: 
\end{itemize}

\subsection{User Characteristic}

As per the definitions given in \ref{sub:Definitions,-acronyms,-and}
the respective applications are targeted to their respective markets:
these consist of any citizen of the town for the customer-side applications,
and of any recognized taxi driver for the taxi-side application. 

Customers do not need specific competences to use the app other than
a basic knowledge of web browsing (customers only) and of their mobile
operating system (both customers and taxi drivers).


\subsection{Domain Assumptions}

Taxi licenses databases


\subsection{Actors and Functionalities}

The platform may have different functional and non-functional requirements
based on the user that is interacting with it. Throughout this document,
the set of these different requirements has been referred to through
the terminology specified in \ref{sub:Definitions,-acronyms,-and}
(e.g. taxi-side application, customer-side application, etc.). 

This section is aimed at further formalizing the definition of those
requirement, based on the different actors that participate in the
use of the platform: 
\begin{itemize}
\item \textbf{A1 Guest}: any person who wishes to start a session with the
platform through one of the user-side applications. A1 may: 

\begin{itemize}
\item Register a profile into the platform 
\item Login into the platform 
\end{itemize}

After completing a login, A1 actors may become either of the two following
types of actor. 

\item \textbf{A2 Customer}: a person as defined in \ref{sub:Definitions,-acronyms,-and}.
Based on what actions A2 completes, they can be further specialized
in:

\begin{itemize}
\item \textbf{A2.1 Logged-in Customer}: an A1 actor who simply logged in
to the platform with a customer profile. A2.1 may: 

\begin{itemize}
\item Issue a taxi request
\item View and edit their profile
\end{itemize}
\item \textbf{A2.2 Passenger}: an A2.1 actor who requested a taxi. A2.2
may:

\begin{itemize}
\item Cancel or edit the previously issued taxi request
\item Issue a taxi report
\end{itemize}
\end{itemize}
\item \textbf{A3 Taxi driver}: a person as defined in \ref{sub:Definitions,-acronyms,-and}.
A3 may:

\begin{itemize}
\item Handle (accept or refuse) a taxi request issued by an A2 actor.
\item Edit their availability status
\item Issue a customer report
\item Notify a problem
\end{itemize}
\end{itemize}

\section{Requirements}


\subsection{Non functional requirements}

The non functional requirements for the platform have been identified
as follows:
\begin{itemize}
\item NF1 Reliability: the system must satisfy the functional requirements
with 100\% of server uptime on the back-end side, and a stable release
of the user-side applications.


The latter is achieved through constant maintenance of the applications'
code bases (outside of project scope), while the former needs to be
designed since the beginning of the project as it relies on specific
hardware infrastructures; these include: 

\item Server farms replication through cloning

\begin{itemize}
\item Local data duplication with RAID controllers
\item Data synchronization between the server farms
\item Periodic memory backup on flash storage to allow for faster recovery
in the event of unexpected system failures
\end{itemize}

These measure should minimize the RTO and RPO parameters making them
tend to zero. 

\item NF2 Performance: \end{itemize}

\end{document}
