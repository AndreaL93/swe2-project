%% LyX 2.1.4 created this file.  For more info, see http://www.lyx.org/.
%% Do not edit unless you really know what you are doing.
\documentclass[12pt,english]{article}
\usepackage{charter}
\renewcommand{\familydefault}{\rmdefault}
\usepackage[T1]{fontenc}
\usepackage[utf8]{luainputenc}
\usepackage{babel}
\usepackage{setspace}
\onehalfspacing
\usepackage[unicode=true,pdfusetitle,
 bookmarks=true,bookmarksnumbered=false,bookmarksopen=false,
 breaklinks=false,pdfborder={0 0 1},backref=false,colorlinks=false]
 {hyperref}

\makeatletter
%%%%%%%%%%%%%%%%%%%%%%%%%%%%%% User specified LaTeX commands.
\usepackage{babel}
\usepackage{babel}

\makeatother

\begin{document}

\section{Overall Description}


\subsection{Product Perspective}

The platform aims at replacing the traditional means of communication
between taxi drivers and customers, and it is therefore designed as
a stand-alone entity that is able to operate with a minimum set of
dependencies towards external systems. 


\subsubsection{Software Interfaces \label{sub:Software-Interfaces}}
\begin{itemize}
\item \textbf{Google Maps API}: used in both the back-end and the taxi-side
applications, this free API is necessary to elaborate routes and display
maps. The system interacts with the API through standard GET requests.

\begin{itemize}
\item Mnemonic: map rendering and route elaboration
\item Documentation: https://developers.google.com/maps/ 
\end{itemize}
\item \textbf{GPS}: this well known system is required by the mobile applications
to get information about the users' position. The interaction happens
through the standard GPS protocol provided by the OS of the users'
devices.

\begin{itemize}
\item Mnemonic: user positions gathering
\item Documentation: refer to OS-specific documentation
\end{itemize}
\item \textbf{Red Hat Enterprise Linux}: operating system run on the back-end
server.

\begin{itemize}
\item Mnemonics: back-end OS 
\item Documentation: https://access.redhat.com/documentation/en/red-hat-enterprise-linux/ 
\end{itemize}
\item \textbf{Apache WebServer}: software bundle run on the back-end, which
is used to allow interaction between the back-end system and the user-side
application. It is used to handle standard HTTP requests from the
customer-side web application and to provide CGI services to all the
user-side applications.

\begin{itemize}
\item Mnemonics: web-server 
\item Documentation: https://httpd.apache.org/docs/2.4/ 
\end{itemize}
\item \textbf{SQLite}: DBMS software component run on the back-end, which
is used to store user data in a non-resource-intensive fashion.

\begin{itemize}
\item Mnemonics: DBMS 
\item Documentation: https://www.sqlite.org/docs.html 
\end{itemize}
\end{itemize}

\subsubsection{User interface}


\subsubsection{Device support\label{sub:Device-support}}

The back-end system is designed to be run exclusively on usual x64
server hardware and may therefore require fine-tuning of its environment,
but needs to be flexible for hardware scaling (such as replication)
in order to ensure its longevity. 

The user-side mobile applications, on the other hand, are designed
to be run on standard SOC architectures like the vast majority that
is found on smartphones and tablets, and take into account the differences
in the mobile operating systems that can be run on the SOC.

Finally, the customer-side web application does not have specific
device requirements, since it relies on a different level of abstraction
(web browser) which is not to be considered as a hardware interface. 


\subsection{Product functions}

The fully developed platform should implement the following high level
functions.
\begin{itemize}
\item \textbf{Customer-side application}: 

\begin{enumerate}
\item[{\textbf{{[}F1{]}}}] Allow users to request a taxi and receive result of request
\item[{\textbf{{[}F2{]}}}] Allow users to manage their profile
\item[{\textbf{{[}F3{]}}}] Allow users to issue taxi reports
\end{enumerate}
\item \textbf{Taxi-side application}:

\begin{enumerate}
\item[{\textbf{{[}F4{]}}}] Allow users to manage incoming requests and send back confirmations 
\item[{\textbf{{[}F5{]}}}] Allow users to issue customer reports
\end{enumerate}
\end{itemize}

\subsection{Actors and related functionalities}

The platform may have different functional and non-functional requirements
based on the user that is interacting with it. Throughout this document,
the set of these different requirements has been referred to with
the terminology specified in \ref{sub:Definitions,-acronyms,-and}
(e.g. taxi-side application, customer-side application, etc.).

This section is aimed at giving a high level description of the platform,
based on the different actors that participate in its use: 
\begin{itemize}
\item[{\textbf{{[}A1{]}}}] \textbf{ Guest}: any person who wishes to start a session with the
platform through one of the user-side applications. {[}A1{]} actors
may:

\begin{itemize}
\item Register a profile into the platform 
\item Log into the platform 
\end{itemize}

After completing a login, {[}A1{]} actors may become either of the
two following types of actor.

\item[{\textbf{{[}A2{]}}}] \textbf{ Customer}: a person as defined in \ref{sub:Definitions,-acronyms,-and}.
Based on what actions {[}A2{]} completes, they can be further specialized
in:

\begin{itemize}
\item[{\textbf{{[}A2.1{]}}}] \textbf{ Logged-in Customer}: an {[}A1{]} actor who logged in to
the platform with a customer profile. {[}A2.1{]} actors may:

\begin{itemize}
\item Issue a taxi request 
\item View and edit their profile 
\end{itemize}
\item[{\textbf{{[}A2.2{]}}}] \textbf{ Passenger}: an {[}A2.1{]} actor who issued a taxi request.
{[}A2.2{]} actors may:

\begin{itemize}
\item Cancel or edit the issued taxi request 
\item Issue a taxi report 
\end{itemize}
\end{itemize}
\item[{\textbf{{[}A3{]}}}] \textbf{ Taxi driver}: a person as defined in \ref{sub:Definitions,-acronyms,-and}.
{[}A3{]} actors may:

\begin{itemize}
\item Handle (accept or refuse) a taxi request issued by an {[}A2{]} actor. 
\item Edit their availability status 
\item Issue a customer report 
\item Notify a technical problem 
\end{itemize}
\end{itemize}

\subsubsection{User characteristics}

Users do not need specific competences to use the app other than a
basic knowledge of web browsing (customers only) and of their mobile
operating system (both customers and taxi drivers).


\subsection{Constraints\label{sub:Constraints}}

The development of the platform will be limited by the following constraints
to which the system must be subject to. These are not listed as part
of the specific requirements of the platform because they stem from
circumstances beyond the designers' control, and are not directly
part of the customer's requests. 
\begin{enumerate}
\item[{\textbf{{[}C1{]}}}] \textbf{Privacy policy}: processing of user data, be it personal
or related to the use of the platform, is to be carried out according
to the national laws and the ways and means of said processing must
be clearly defined and available to the users at any time. 
\item[{\textbf{{[}C2{]}}}] \textbf{Taxi licenses}: the use of the taxi-side application and
the related access to the platform, is intended exclusively for taxi
drivers officially recognized by the town's appropriate taxi organizations.
\item[{\textbf{{[}C3{]}}}] \textbf{Data integrity and data security}: users' personal data must
be stored according to specific standards that guarantee their integrity
and security. Further analysis of this constraint will be carried
out in the following sections of the document. 
\item[{\textbf{{[}C4{]}}}] \textbf{Legal waiver of responsibility}: due to the nature of the
platform's domain, the use of the platform can be associated to car
accidents and other related possible injuries; users are therefore
required to accept a legal disclaimer, in order to use the platform.
\end{enumerate}

\subsection{Domain Assumptions}

Some domain assumption have to be introduced in order to further define
the scope of the project, and to clearly state under which assumptions
the development should proceed in those areas that are not explicitly
covered by the initial customer's specification. 

The mentioned assumptions are the following:
\begin{enumerate}
\item[{\textbf{{[}AS1{]}}}] It is assumed that the user-side devices are capable of running the
intended user-side applications, according to the descriptions in
section \ref{sub:Device-support}. Hardware or software compatibilities
will not be discussed further.
\item[{\textbf{{[}AS2{]}}}] It is assumed that a public database exists, which can be exploited
to perform checks on the official status of taxi drivers in order
to ensure the satisfaction of constraint {[}C2{]} of section \ref{sub:Constraints}.
\item[{\textbf{{[}AS3{]}}}] It is assumed that the number of users in the first year of operation
will not exceed the platform's capacity, and modifications to the
platform will not be needed. Later modifications are not discussed
in the present document. 
\item[{\textbf{{[}AS4{]}}}] It is assumed that any pre-existing taxi reservation service shall
not be integrated with the platform.\end{enumerate}

\end{document}
