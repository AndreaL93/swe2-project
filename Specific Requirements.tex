%% LyX 2.1.4 created this file.  For more info, see http://www.lyx.org/.
%% Do not edit unless you really know what you are doing.
\documentclass[english]{article}
\usepackage[T1]{fontenc}
\usepackage[utf8]{inputenc}
\usepackage{array}
\usepackage{longtable}
\usepackage{graphicx}

\makeatletter

%%%%%%%%%%%%%%%%%%%%%%%%%%%%%% LyX specific LaTeX commands.
%% Because html converters don't know tabularnewline
\providecommand{\tabularnewline}{\\}

%%%%%%%%%%%%%%%%%%%%%%%%%%%%%% User specified LaTeX commands.
\usepackage{array}




\providecommand{\tabularnewline}{\\}

\usepackage{babel}
\graphicspath{{images/}}




\usepackage{babel}

\makeatother

\usepackage{babel}
\begin{document}

\section{One }


\section{Two}


\section{Specific Requirements}

This section of the document is dedicated at giving an in-depth description
of the platform's requirements, and is to be kept as reference during
all future phases of development.


\subsection{External interfaces\label{sub:External-interfaces}}

Being MyTaxiService a fully service-oriented platform, its only external
interfaces must be those reserved for the final users; there is no
need to design specific maintenance access to the back-end system
as this is already fully standardized and does not need specific functionalities
other than the usual system administration tools.

As briefly described in section \ref{sub:User-interface}, the main
principle that must guide the design of the external interfaces of
the platform is that of business identity continuity. This section
contains a set of design mock-ups that are to be kept as reference
during the development of the user interfaces.

//MOCKUPS <\textcompwordmark{}<\textcompwordmark{}<\textcompwordmark{}<\textcompwordmark{}<\textcompwordmark{}<\textcompwordmark{}<\textcompwordmark{}<\textcompwordmark{}<\textcompwordmark{}<\textcompwordmark{}<\textcompwordmark{}<\textcompwordmark{}<\textcompwordmark{}<\textcompwordmark{}<\textcompwordmark{}<\textcompwordmark{}<\textcompwordmark{}<\textcompwordmark{}<\textcompwordmark{}<\textcompwordmark{}<\textcompwordmark{}<\textcompwordmark{}<\textcompwordmark{}<\textcompwordmark{}<\textcompwordmark{}<\textcompwordmark{}<\textcompwordmark{}<\textcompwordmark{}<\textcompwordmark{}<\textcompwordmark{}<\textcompwordmark{}<\textcompwordmark{}<\textcompwordmark{}<\textcompwordmark{}<\textcompwordmark{}<\textcompwordmark{}<\textcompwordmark{}<\textcompwordmark{}<\textcompwordmark{}<\textcompwordmark{}<\textcompwordmark{}<\textcompwordmark{}<\textcompwordmark{}<\textcompwordmark{}<\textcompwordmark{}<\textcompwordmark{}<\textcompwordmark{}<\textcompwordmark{}<


\subsection{Functional requirements}

The following set of functional requirements must be kept as official
reference for all further phases of development of the platform. 

These requirements are meant as a more in-depth description of the
main system functionalities listed in section \ref{sub:Product-functions},
and must be considered alongside the domain assumptions in section
\ref{sub:Domain-Assumptions}. 


\subsubsection{Back-end requirements\label{sub:Back-end-requirements}}
\begin{enumerate}
\item The system must act as the central point of communication between
taxi drivers and customers (functions F1 and F4).

\begin{enumerate}
\item The system must forward the customers' requests to taxi drivers.
\item The system must forward the taxi drivers' replies (in response to
requests of point 1.(a)) to customers.
\item The system must manage the exceptions that may happen in the flow
of events (see section \ref{sub:Flow-of-events} for more specific
description of these exceptions)
\end{enumerate}
\item The system must manage the customers taxi reservation requests and
forward them to taxi drivers when necessary. (functions F1 and F4).
\item The system must manage the taxi queues associated to the different
zones of the city (function F9).

\begin{enumerate}
\item The application must keep track of the availability statuses of all
taxi drivers (see also requirement 3 of section \ref{sub:Taxi-side-application-specific})
and manage the queues accordingly. 
\end{enumerate}
\item The system must manage the operational databases.

\begin{enumerate}
\item The system must store user operation critical data (username, password,
e-mail address, name, date of birth).
\item The system must keep track of changes in user information when prompted
(functions F2 and F5).
\item The system must store user records submitted through the \textit{Report}
functions (functions F3 and F6).
\end{enumerate}
\item The system must perform the necessary validation and consistency checks
on any data it handles in order to ensure the functional and non functional
requirements listed in the following sections. 
\end{enumerate}

\subsubsection{Generic user application requirements}
\begin{enumerate}
\item The application must act as external interface between users and the
back-end.

\begin{enumerate}
\item The application must enable users to register into the platform.
\item The application must enable users to log into the platform. 
\item The application must provide users with an interface to manage their
personal information (functions F2 and F5).
\end{enumerate}
\item The application must display confirmations and error messages forwarded
by the back-end (see requirement 1 of section \ref{sub:Back-end-requirements},
and section \ref{sub:Flow-of-events}).
\item The application must provide all the implementation-specific requirements
listed in the following sections. 
\end{enumerate}

\subsubsection{Taxi-side application specific requirements\label{sub:Taxi-side-application-specific}}
\begin{enumerate}
\item The application must notify the taxi driver when a request from a
customer is forwarded by the back-end (function F4). 
\item The application must show the taxi driver's position in their local
queue. 
\item The application must enable the taxi driver to toggle their availability
status between \textit{available }and \textit{unavailable}. 
\item The application must enable the taxi driver to submit \textit{Customer}
reports (function F6).

\begin{enumerate}
\item The application must require information about the ride if the information
is not directly deductible from the known data. 
\end{enumerate}
\item The application must enable the taxi driver to submit \textit{Technical
problem} reports (function F7). 

\begin{enumerate}
\item The application must require information about the technical problem
before submitting the report to the system. 
\item The application must require the taxi driver to state whether they
want a replacement to complete the ride (if the technical problem
happens when a passenger is on board) before submitting the report
to the system.
\end{enumerate}
\end{enumerate}

\subsubsection{Customer-side application specific requirements\label{sub:Customer-side-application-specif}}
\begin{enumerate}
\item The application must enable the customer to request the services of
a taxi. (function F1).

\begin{enumerate}
\item The application must enable the customer to request a taxi in the
very short term (service is exploited as soon as the taxi is available). 
\item The application must enable the customer to reserve a taxi in advance.

\begin{enumerate}
\item The application must require the customer's destination before submitting
this kind of request to the system. 
\end{enumerate}
\item The application must enable the customer to select the \textit{Taxi
sharing} option (it allows users with similar routes at the same hour
to share the taxi and split the fee)

\begin{enumerate}
\item The application must require the customer's starting point and destination
before submitting this kind of request to the system. 
\end{enumerate}
\item The application must enable the customer to select both options of
points 1.(b) and 1.(c) of this section at the same time. 
\end{enumerate}
\item The application must enable the customer to submit a \textit{Taxi
}reports (function F3).

\begin{enumerate}
\item The application must require information about the ride if the information
is not directly deductible from the known data. 
\end{enumerate}
\end{enumerate}

\subsubsection{Functional constraints}
\begin{enumerate}
\item Taxi reservations must can only be issued at least 2 hours before
the requested time of the ride (function F1, associated to req. \ref{sub:Customer-side-application-specif}
1.(b)). 
\item Requests associated to taxi reservations must be forwarded to taxi
drivers exactly 10 minutes before the requested time (function F9,
associated to req. \ref{sub:Customer-side-application-specif} 1.(b)). 
\item \textit{Taxi} and \textit{Customer }reports must be accepted only
on condition that the ride happened in a 24 hours time frame from
the submission request functions (functions F3 and F6). 
\end{enumerate}

\subsection{Use Cases}

\includegraphics[width=1\textwidth]{\string"Use_cases_and_Sequence_diagram/SequenceDiagrams svg/svg/Model1__UseCaseDiagram1_17\string".eps}


\subsection{Logical database requirements}

The majority of data processed by the DMBS consists of Java \textit{String
}and \textit{Date }objects. Images and complex objects might be stored
in the database, too, and must therefore be considered during the
design of the database. 

The majority of stored data is not frequently accessed as most of
the operational variables are strictly mutable (users' positions,
queues, non-booked rides) and therefore not in need to be stored. 

Stored data might consist of stable personal information of the users
and, while not immutable, it can be considered as rarely accessed. 

A deeper insight on the data classes and operational variables is
found in the next sections of this document, so it is advised to keep
this section only as a general reference and to approach the design
of the database by looking at the more specific descriptions given
there. 


\subsubsection{Class diagram }

\includegraphics[width=1\textwidth]{\string"Use_cases_and_Sequence_diagram/SequenceDiagrams svg/svg/Model2__ClassDiagram1_20\string".eps}


\subsubsection{Entity Behaviors (State chart diagrams)}
\begin{itemize}
\item[{\textbf{{[}SC1{]}}}] User
\end{itemize}
\includegraphics[width=1\textwidth,height=1\textheight]{\string"Use_cases_and_Sequence_diagram/SequenceDiagrams svg/svg/StateMachine3__User_23\string".eps}
\begin{itemize}
\item[{\textbf{{[}SC2{]}}}] Ride
\end{itemize}
\includegraphics[width=1\textwidth,height=1\textheight]{\string"Use_cases_and_Sequence_diagram/SequenceDiagrams svg/svg/StateMachine2__Ride_22\string".eps}
\begin{itemize}
\item[{\textbf{{[}SC3{]}}}] Queue Manager
\end{itemize}
\includegraphics[width=1\textwidth,height=1\textheight]{\string"Use_cases_and_Sequence_diagram/SequenceDiagrams svg/svg/StateMachine1__Queue Manager _21\string".eps}


\subsection{Scenarios}

To help the reader understand the above stated requirements, a brief
description of how a use case might look like in the real world is
given below.

In the examples, Adam, Michelle and Joanne are customers who intend
to request a taxi and Hector, Monica, Jim and Samuel are taxi drivers
of the town.


\subsubsection{Sign up}

Adam has just downloaded the customer-side app and wants to sign up
into the platform. He requests the customer registration page, fills
the form and submits the request to the system. If Adam's e-mail and
username are unique, the system gives Adam a confirmation of the success
of the operation and redirects Adam to the login page; otherwise,
an error message is displayed on Adam's phone.


\subsubsection{Login}

Adam, now registered, inserts the username and password in the login
form and clicks the login button; the system checks the information
and, if the username-password combination is correct, redirects Adam
to his own user profile page; otherwise, an error message is displayed
on Adam's phone.


\subsubsection{Available}

Hector, already logged into the platform, starts his working by day
opening his taxi-side application and communicating his availability
to the system. The system updates the taxi queue in Hector's zone
and sends Hector a notification with his position in the queue.


\subsubsection{Taxi request\label{sub:Taxi-request}}

Adam, now logged into the system, wants to book a taxi to go home.
He opens the taxi request page on the app, and requests a taxi. The
system forwards Adam's request to the queue associated with Adam's
position, and Hector, which is the first taxi driver in the queue,
is notified with the request.

Unfortunately, Hector has now decided to take a break and does not
want to take charge of this ride; he refuses Adam's request by tapping
a button on the app, and the system forwards the request to Monica,
the taxi driver immediately after Hector in the queue.

As she accepts Adam's request, Adam receives a notification on the
app with the estimated waiting time.


\subsubsection{Book a Taxi\label{sub:Book-a-Taxi}}

While on Monica's taxi, Adam wants to book a taxi for that evening
at 6 PM, in order to go to the cinema. He opens the \textit{Taxi request}
page of the app, and fills and submits the request form.

The system checks the information (sending eventual error notifications
back to Adam) and forwards Adam's request to Jim, by using a specific
selection algorithm over taxi drivers in the queue associated to Adam's
zone.

Ten minutes before 6PM the system will forward Adam's request to the
first taxi in the queue at that time, and similarly to the previous
scenario a taxi will be assigned to Adam. 


\subsubsection{Car sharing}

Michelle and Joanne live in the same neighborhood, and they both decide
to go see a fair on the other side of the Town. Since they are both
short on money, after opening the \textit{Taxi request }page of the
app they both check the \textit{car sharing }option; after checking
the option, the apps automatically adds a field in the reservation
form in which they must specify their intended destination; they then
submit their requests. 

The system performs a check on Michelle and Joanne's requests and,
since their routes match, it automatically assign them to the same
taxi. 

Based on whether they decided to book the taxi or simply request it,
the taxi will be chosen similarly to scenarios \ref{sub:Book-a-Taxi}
or \ref{sub:Taxi-request} respectively. 


\subsubsection{Manage \textit{Reserve Taxi} Request}

Later that day, Adam browses the platform's website from his laptop's
browser, and opens the \textit{Manage taxi request} page to change
the booking time from 6PM to 7PM. 

The system checks whether Adam's request is acceptable (there must
be at least two hours between the current time and the requested time),
and possibly modifies to time on which to forward the request to the
queue.


\subsubsection{Report Taxi}

Jim picks Adam up at 7PM. During the ride Jim lights up a cigarette
and is unreasonably rude towards Adam.

Adam opens the \textit{Report taxi} page on the app, to file a complaint
about Jim's behavior. The system updates Jim's record with the new
report and confirms the success of the operation to Adam.


\subsubsection{Report Customer}

After the ride, Adam is annoyed by the behavior of Jim and refuses
to pay for the ride.

Jim opens the \textit{Report user} page, fills the complaint form
and submits it to the system. The system updates Adam's record with
the new report and confirms the success of the operation to Jim.


\subsubsection{Manage Personal Information}

Joanne has opened a new main email account.

She opens her profile page from the app, clicks on the \textit{edit}
button and changes her email address to match the new one; she then
submits the new information.

The system performs a check on the information, updates Joanne's profile
and notifies the success of the operation to Joanne.


\subsubsection{Report Problem}

During a ride, Hector has a problem with his taxi's engine and can't
bring Joanne to her destination.

Through the \textit{Report problem} page of the app, he notifies the
problem to the system by filling the form and submitting. The system
acknowledges the report and asks Hector if he'll be needing a new
taxi; Hector confirms, and the system forwards his request to Samuel,
who is the first taxi driver in Hector and Joanne's current zone.


\subsection{Flow of events\label{sub:Flow-of-events}}


\subsubsection{Sign-up}

\begin{tabular}{lp{8cm}}
\hline 
Actors  & \ref{-Guest:-any}Guest\tabularnewline
\hline 
Preconditions  & The guest is not registered into the system.\tabularnewline
\hline 
Execution Flow  & \begin{enumerate}
\item The guest requests the registration page.
\item The guest fills the registration form and submits the request.
\item The system checks the uniqueness of the username and e-mail.
\item The system creates the customer (or taxi driver) profile .
\item The system sends the confirmation to the guest.\end{enumerate}
\tabularnewline
\hline 
Postconditions  & The guest is now a registered user.\tabularnewline
\hline 
Exceptions  & The e-mail or username are not unique or, in the case of a taxi driver
sign-up, the license is not valid: an error message is shown. \tabularnewline
\end{tabular}

\includegraphics[width=1\textwidth]{\string"Use_cases_and_Sequence_diagram/SequenceDiagrams svg/svg/Collaboration1__Interaction1__Signup_2\string".eps}


\subsubsection{Login}

\begin{tabular}{lp{8cm}}
\hline 
Actors  & Guest \tabularnewline
\hline 
Preconditions  & The guest is already registered into the system.\tabularnewline
\hline 
Execution Flow  & \begin{enumerate}
\item The guest requests the login page.
\item The guest fills the form and submits the request.
\item The system checks the username and password.
\item The system sends a login confirmation.
\item The guest is logged into the system.
\item The guest is redirected to the user main page.\end{enumerate}
\tabularnewline
\hline 
Postconditions  & The guest is now a logged-in user.\tabularnewline
\hline 
Exceptions  & The username-password combination is incorrect, so the guest cannot
log in: an error message is shown.\tabularnewline
\end{tabular}

\includegraphics[height=1\textheight]{\string"Use_cases_and_Sequence_diagram/SequenceDiagrams svg/svg/Collaboration2__Interaction1__Login_3\string".eps}


\subsubsection{Available}

\begin{tabular}{lp{8cm}}
\hline 
Actors  & Taxi driver \tabularnewline
\hline 
Preconditions  & The taxi driver is logged into the system.\tabularnewline
\hline 
Execution Flow  & \begin{enumerate}
\item The taxi driver opens the app.
\item The taxi driver changes their availability by tapping a button: a
request is sent to the system. 
\item The system updates the queue. 
\item The system returns a confirmation to the taxi driver. \end{enumerate}
\tabularnewline
\hline 
Postconditions  & The taxi driver has now changed their availability.\tabularnewline
\hline 
Exceptions  & \begin{itemize}
\item The taxi driver is located in a invalid zone and they try to become
\textit{available}: an error message is shown. \end{itemize}
\tabularnewline
\end{tabular}

\includegraphics[width=1\textwidth]{\string"Use_cases_and_Sequence_diagram/SequenceDiagrams svg/svg/Collaboration10__Interaction1__Available_11\string".eps}


\subsubsection{Taxi Request}

\begin{longtable}[l]{l>{\raggedright}p{8cm}}
\hline 
Actors  & Customer, Taxi driver \tabularnewline
\hline 
Preconditions  & Both users must be logged in, the taxi driver must be available.\tabularnewline
\hline 
Execution Flow  & \begin{enumerate}
\item The customer requests the \textit{Taxi request} page 
\item The customer is shown a form to fill with optional information: they
can choose to book a taxi and to enable the \textit{Taxi sharing}
option. 
\item The customer fills the request form according to their preference
and sends the information to the system.
\item Based on the type of request that the customer issued, the system
generates a request with all the needed data and sends it to the first
taxi driver in the right local queue, at the right time. 
\item The taxi driver can either ignore the request or accept it. 
\item If the taxi driver accepts the request, the system notifies to the
customer the incoming taxi (with an approximate ETA) and changes the
availability of the taxi driver; otherwise, the system puts the taxi
driver at the end of the queue and forwards the request to the next
first taxi driver of the queue. 
\item If the issued request was a booking request or a request with \textit{Taxi
sharing }enabled, the system calculates the estimated fee for the
passenger and adds it to the notification sent to the user. \end{enumerate}
\tabularnewline
\newpage
\hline 
Postconditions  & If the request is accepted by a taxi driver, the customer is now a
passenger.\tabularnewline
\hline 
Exceptions  & \begin{itemize}
\item The customer provides incorrect information in the request form: an
error notification is shown. 
\item No taxis are available: the system notifies so to the user.
\item The customer is not in a valid position (\emph{e.g. outside the town}):
an error notification is shown. \end{itemize}
\tabularnewline
\end{longtable}

\includegraphics[width=1\textwidth]{\string"Use_cases_and_Sequence_diagram/SequenceDiagrams svg/svg/Collaboration3__Interaction1__Taxi Request_4\string".eps}


\subsubsection{Manage \textit{Reserve Taxi} Request}

\begin{tabular}{lp{8cm}}
\hline 
Actors  & Customer\tabularnewline
\hline 
Preconditions  & \begin{itemize}
\item The customer must have reserved a taxi. 
\item The customer must be logged in.\end{itemize}
\tabularnewline
\hline 
Execution Flow  & \begin{enumerate}
\item The customer requests the \textit{Taxi request management} page.
\item The customer can modify the request by filling a\textit{ }form and
submitting it, or delete the request by tapping (or clicking) a button. 
\item The system modifies the request and returns a confirmation to the
passenger. 
\item The request is forwarded to the right local queue accordingly; if
the user canceled their request, the request is not sent. \end{enumerate}
\tabularnewline
\hline 
Postconditions  & \begin{itemize}
\item If the customer chooses to modify the request, the request is updated.
\item If the customer chooses to delete the request, the request is canceled.\end{itemize}
\tabularnewline
\hline 
Exceptions  & \begin{itemize}
\item The customer provides incorrect information in the \textit{Modify
request} form: an error message is shown. 
\item The customer tried to cancel the request when the request has already
been forwarded to the taxi driver: an error message is shown and the
modification is not allowed.\end{itemize}
\tabularnewline
\end{tabular}

\includegraphics[width=1\textwidth]{\string"Use_cases_and_Sequence_diagram/SequenceDiagrams svg/svg/Collaboration15__Interaction1__Manage Taxi Request_19\string".eps}


\subsubsection{Report Taxi}

\begin{tabular}{lp{8cm}}
\hline 
Actors  & Passenger \tabularnewline
\hline 
Preconditions  & \begin{itemize}
\item The interaction between the passenger and the taxi driver must have
happened at most 24 hours before.
\item The passenger must be logged in.\end{itemize}
\tabularnewline
\hline 
Execution Flow  & \begin{enumerate}
\item The passenger requests the \textit{Report taxi} page.
\item The passenger fills the form and submits the report.
\item The system checks the submitted data.
\item The system updates the taxi driver's record.
\item The system notifies to the passenger the success of the operation.\end{enumerate}
\tabularnewline
\hline 
Postconditions  & The taxi driver is reported by the passenger.\tabularnewline
\hline 
Exceptions  & The passenger provides incorrect information in the report form: an
error message is shown. \tabularnewline
\end{tabular}

\includegraphics[width=1\textwidth]{\string"Use_cases_and_Sequence_diagram/SequenceDiagrams svg/svg/Collaboration7__Interaction1__Report Taxi_8\string".eps}


\subsubsection{Report Customer}

\begin{tabular}{lp{8cm}}
\hline 
Actors  & Taxi driver \tabularnewline
\hline 
Preconditions  & \begin{itemize}
\item The interaction between the passenger and the taxi driver must have
happened at most 24 hours before. 
\item The taxi driver must be logged in.\end{itemize}
\tabularnewline
\hline 
Execution Flow  & \begin{enumerate}
\item The taxi driver requests the \textit{Report customer} page.
\item The taxi driver fills the form and submits the report.
\item The system checks the submitted data.
\item The system updates the custormer's record.
\item The system notifies to the taxi driver the success of the operation.\end{enumerate}
\tabularnewline
\hline 
Postconditions  & The customer is reported by the taxi driver.\tabularnewline
\hline 
Exceptions  & The taxi driver provides incorrect information in the report form:
an error message is shown. \tabularnewline
\end{tabular}

\includegraphics[width=1\textwidth]{\string"Use_cases_and_Sequence_diagram/SequenceDiagrams svg/svg/Collaboration13__Interaction1__Report Customer_14\string".eps}


\subsubsection{Manage Personal Information}

\begin{tabular}{lp{8cm}}
\hline 
Actors  & Customer or Taxi driver \tabularnewline
\hline 
Preconditions  & The user must be logged in.\tabularnewline
\hline 
Execution Flow  & \begin{enumerate}
\item The user requests their profile page.
\item The user's personal information is shown on the user's application.
\item The user can begin editing their profile information by tapping (or
clicking) on the \textit{Edit} button.
\item The user edits their information and submits the changes to the system.
\item The system performs a check on the new information.
\item If the information is correct, a confirmation is sent back to the
user.\end{enumerate}
\tabularnewline
\hline 
Postconditions  & The user profile information is changed.\tabularnewline
\hline 
Exceptions  & The user provides incorrect information: an error message is shown. \tabularnewline
\end{tabular}

\includegraphics[width=1\textwidth]{\string"Use_cases_and_Sequence_diagram/SequenceDiagrams svg/svg/Collaboration9__Interaction1__Manage your Personal Info_10\string".eps}


\subsubsection{Report Problem}

\begin{tabular}{lp{8cm}}
\hline 
Actors  & Taxi driver \tabularnewline
\hline 
Preconditions  & The taxi driver must be logged in\tabularnewline
\hline 
Execution Flow  & \begin{enumerate}
\item The taxi driver requests the \textit{Report problem} page.
\item The taxi driver fills the form and submits the information regarding
a technical problem that they are experiencing.
\item If the taxi driver has a passenger on board, they can request another
taxi to drive the passenger to their destination.
\item If the taxi driver requests another taxi the system looks for an available
taxi driver, with the usual procedure. 
\item If a taxi driver accepts the request, a confirmation is sent to the
driver who is submitting the report.\end{enumerate}
\tabularnewline
\hline 
Postconditions  & The technical problem is reported to the system \tabularnewline
\hline 
Exceptions  & \begin{itemize}
\item The taxi driver is located in a invalid zone: an error message is
shown.
\item The taxi driver requests a second taxi, but no available taxi is found:
an error message is shown to the user. \end{itemize}
\tabularnewline
\end{tabular}

\includegraphics[width=1\textwidth]{\string"Use_cases_and_Sequence_diagram/SequenceDiagrams svg/svg/Collaboration14__Interaction1__Notify Problem_18\string".eps}


\subsection{Performance requirements}

Some indicative non functional requirements regarding performance
have been identified as follows. 
\begin{enumerate}
\item The platform must support a number users equals to 3 times the number
of registered taxi drivers in the Town. Estimates must be calculated
each year and modifications to the infrastructure must be made accordingly. 
\item The system must process 99\% of requests in less than 5 seconds.
\item The system must support parallel processing of the request to a degree
proportional to at least 25\% of the average number of requests. 
\item Management of the database must be transparent to the user, which
must have the impression of a continuous interaction with the system. 
\item Estimates of the costs of the rides must be precise with a 10\% error
margin. 
\item Estimates of the taxi drivers' ETA must be precise with a 10\% error
margin.
\end{enumerate}

\subsection{Availability and Reliability Requirements}

Since MyTaxiService is a service-oriented platform, its reliability
parameters directly relate to its availability parameters. The platform's
ability to function under the stated conditions is indeed its ability
to respond to users' requests at any given time, hence the strict
relation between the two. It has been decided to treat the two aspect
as one, and the related non functional requirements are listed in
this section.
\begin{enumerate}
\item The platform's services must be available to the users 24/7.
\item The RTO parameter must be kept at minimal levels (less than 1 minute)
at any given time.

\begin{enumerate}
\item Mission critical data must be locally mirrored on fast hardware (e.g.
stored in RAID1 arrays with flash storage).
\end{enumerate}
\item The RPO parameter must be kept at minimal levels (less than 10 seconds)
at any given time.

\begin{enumerate}
\item Any data must be locally stored in a 10 second time frame from its
creation.
\item Any locally stored data must be locally and remotely mirrored in a
1 minute time frame from its memorization. 
\end{enumerate}
\item Data integrity checks must be periodically performed between the main
data storage unit and the secondary backups, in order to ensure the
success of disaster recovery operations.
\item The implementation of the platform must prefer the absence of service
to an incorrect or unsound one. 

\begin{enumerate}
\item No data exchanges must happen during the disaster recovery operations.
\item Data stored in memory in the event of a system failure or security
breach must be considered corrupt and no attempts must be made at
recovering it. 
\end{enumerate}
\end{enumerate}

\subsection{Security Requirements}

The following non functional requirements cover the security aspects
of the platform in order, among other reasons, to satisfy the C3 constraint
in section \ref{sub:Constraints}.
\begin{enumerate}
\item Access to the user data through the intended applications must be
password protected.

\begin{enumerate}
\item A ban system must exists to prevent brute-forcing of the users' passwords.
\end{enumerate}
\item Sensitive user data (like passwords) must be stored under at least
one encryption layer, after having been \textit{salted}. This applies
to secondary storage, too. 

\begin{enumerate}
\item Decryption of the above mentioned data must happen exclusively at
runtime and the \textit{cleartext }information must never be sent
through any communication channels. 
\end{enumerate}
\item Operations on the platform must be performed exclusively by logged
users (with the exception of the guest registration).
\item HTTP data exchanges between the back-end and the user-side applications
must be encrypted with a recognized SSL certificate (HTTPS protocol).
\item Access to the back-end system must be protected both via hardware
and software means.

\begin{enumerate}
\item A physical firewall must exists between the Internet and the back-end
main router.
\item Access to the system must be enabled via IP address whitelisting,
rather than blacklisting.
\item Root login must be disabled for remote sessions.
\item Password login must be disabled and signed PKA must be enforced, for
any type of session.
\item Access logs must be kept, backed up, and regularly analyzed.
\end{enumerate}
\item Mission critical data must be stored with a particular attention to
data integrity. 
\end{enumerate}

\subsection{Maintainability Requirements}

The following non functional requirements regarding the maintainability
of the codebase are meant as a small guideline for programmers and
designers in the development phase.
\begin{enumerate}
\item The codebase for all developed software must be highly modular to
facilitate possible changes in the platform's functions and possible
integration with other systems; this applies especially to the back-end
modules.
\item The codebase for all developed software must be thoroughly documented
with both in-code comments and official documentation, in order to
facilitate a possible outsourcing of the maintenance phase. 
\end{enumerate}

\subsection{Portability Requirements}

The following non functional requirements consider technical details
of the platform's implementation in order to analyze its portability
requirements. 

When seen as a whole, the platform consists mainly of its user-side
applications, and the back-end accounts for about 25\% of the codebase;
nonetheless, since the user-side applications are strictly OS dependent,
as specified in section \ref{sub:User-interface}, portability is
an issue which has to be tackled in back-end development, in order
to keep costs to a minimum in the case of possible changes in the
platform (e.g. an integration with a pre-existsing system). Therefore:
\begin{enumerate}
\item The back-end software must be developed in Java Enterprise Edition.
\item Integration with support modules must happen through JEE libraries.
\item Any system related calls, communication protocols and thread related
calls in the back-end must be OS independent (the use of wrapper libraries
is encouraged over a case-by-case analysis).\end{enumerate}

\end{document}
