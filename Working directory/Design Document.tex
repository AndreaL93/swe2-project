%% LyX 2.1.4 created this file.  For more info, see http://www.lyx.org/.
%% Do not edit unless you really know what you are doing.
\documentclass[english]{article}
\usepackage{charter}
\renewcommand{\familydefault}{\rmdefault}
\usepackage[T1]{fontenc}
\usepackage[latin9]{inputenc}
\usepackage{fancyhdr}
\pagestyle{fancy}
\usepackage{babel}
\usepackage{graphicx}
\usepackage{setspace}
\onehalfspacing
\usepackage[unicode=true,pdfusetitle,
 bookmarks=true,bookmarksnumbered=false,bookmarksopen=false,
 breaklinks=false,pdfborder={0 0 0},backref=false,colorlinks=false]
 {hyperref}

\makeatletter

%%%%%%%%%%%%%%%%%%%%%%%%%%%%%% LyX specific LaTeX commands.
%% Because html converters don't know tabularnewline
\providecommand{\tabularnewline}{\\}

%%%%%%%%%%%%%%%%%%%%%%%%%%%%%% User specified LaTeX commands.
\usepackage{babel}
\usepackage{babel}
\usepackage{babel}
\usepackage{babel}






\usepackage{listings}
\renewcommand{\lstlistingname}{Listing}





\usepackage{listings}
\renewcommand{\lstlistingname}{Listing}



\usepackage{listings}
\renewcommand{\lstlistingname}{Listing}

\makeatother

\begin{document}

\title{Design Document}

\maketitle
\begin{center}
\includegraphics[height=0.3\textheight]{polimiLogo}
\par\end{center}

\begin{center}
Daniele Grattarola (Mat. 853101)
\par\end{center}

\begin{center}
Ilyas Inajjar (Mat. 790009) 
\par\end{center}

\begin{center}
Andrea Lui (Mat. 850680)
\par\end{center}


\title{\pagebreak{}}

\tableofcontents{}

\pagebreak{}


\section{Introduction}


\subsection{Purpose}


\subsection{Scope}


\subsection{Definitions, acronyms, and abbreviations \label{sub:Definitions,-acronyms,-and}}

Throughout this document, the following definitions will be applied
without further explanations: 
\begin{itemize}
\item \textbf{Platform}: the set of software applications and hardware infrastructure
that make up the MyTaxiService service; these include the back-end
server software, the web application and the mobile application used
by the customers, the mobile application used by the taxi drivers,
and all the necessary hardware needed to run the mentioned software
and any needed support software. 
\item \textbf{System}: any subset of software components of the platform. 
\item \textbf{Back-end}: the software run on the back-end server of the
platform which is used to handle the communication between the user
applications. The term also addresses all the necessary software components
that are needed to store data, perform calculations and manage the
hardware (e.g. an operating system). 
\item \textbf{Customer-side application}: software run on a personal device
which is used to send taxi requests to the system and to handle the
system's replies. It is designed to be used by customers (see below)
and can either be a mobile application (run on a smartphone or tablet)
or a web application (run on any personal device through an Internet
browser). 
\item \textbf{Taxi-side application}: software run on a personal device
which is used to manage taxi requests forwarded by the system and
to reply to the system with information on how to handle the requests.
It is designed to be used by taxi drivers (see below) and is a mobile
application (run on a smartphone or tablet). 
\item \textbf{User-side application}: the set of customer-side and taxi-side
applications 
\item \textbf{Taxi driver}: the owner of a taxi license in Pallet Town,
who uses the taxi-side application to interact with the platform. 
\item \textbf{Customer}: a person which intends to exploit the taxi service
of the town, and who uses the user-side applications to interact with
the platform. 
\item \textbf{User}: any customer or taxi driver who uses a user-side application. 
\end{itemize}
The following acronyms will also be used in place of the extended
form: 
\begin{itemize}
\item \textbf{DD: }Design Document
\item \textbf{RASD}: Requirement Analysis and Specification Document 
\item \textbf{CGI}: Common Gateway Interface 
\item \textbf{DBMS}: Data Base Management System 
\item \textbf{RTO}: Recovery Time Objective 
\item \textbf{RPO}: Recovery Point Objective 
\item \textbf{SOC}: System On a Chip 
\end{itemize}
The following convention will be used to refer to different items
in the document:
\begin{itemize}
\item \textbf{sec. / secs.}: section / sections
\item \textbf{req.}: requirement.
\end{itemize}
A typical use of the aforementioned abbreviation would be in the form
``element Xx, sec. x.x.x'' (e.g. \textit{req. 1, sec. 1.3} - if
this section contained a numbered requirement with index 1).

One last observation is to be done regarding the use of the singular
\textit{they}, which will be used to refer to single persons throughout
the whole document. 


\subsection{Overview}

The presented RASD is divided in sections and structured as follows: 
\begin{itemize}
\item \textbf{Section 1 - Introduction}: contains support information to
better understand the presented document and provides a brief introduction
of the project, with its scope and general goals. 
\item \textbf{Section 2 - Architectural Design}: contains a high level descriptions
of the platform to be created, with its functions, conceptual structure,
relations with the outside world and design constraints. 
\item \textbf{Section 3 - Algorithm Design}: Focus on the definition of
the most relevant algorithmic part of your project. 
\item \textbf{Section 4 - User Interface Design}: Provide an overview on
how the user interface(s) of your system will look like. If you have
included this part in the RASD, you can simply refer to what you have
already done, possibly, providing here some extensions if applicable.
\item \textbf{Section 5 - Requirements} \textbf{Traceability}: Explain how
the requirements you have defined in the RASD map into the design
elements that you have defined in this document.
\item \textbf{Section 6 - References: }Additional comments about this document.
\end{itemize}

\section{Architectural Design}


\subsection{Overview \label{sub:Architectural-Overview}}


\subsection{High level components and their interaction\label{sub:High-Level-Components}}


\subsection{Component view\label{sub:Component-View}}


\subsection{Deployment view\label{sub:Deployment-View}}


\subsection{Runtime view\label{sub:Runtime-View}}


\subsection{Component interfaces\label{sub:Component-Interfaces}}


\subsection{Selected architerctural styles and patterns\label{sub:Selected-Architectural}}


\subsection{Other design decisions\label{sub:Other-Design}}


\section{Algorithm Design}


\section{User Interface Design}


\section{Requirements Traceability}


\section{References}

The production of this document has been a joint effort of all the
authors, with a fair distribution of the mansions which caused each
member of the group to work on all the parts of the document. 

The production has been carried out between 16/10/2015 and 6/11/2015
for a total time expense of: 
\begin{itemize}
\item \textbf{Group work}: hours
\item \textbf{Individual work}:


\begin{tabular}{|c|c|}
\hline 
Daniele Grattarola (Mat. 853101) &  hours\tabularnewline
\hline 
Ilyas Inajjar (Mat. 790009)  &  hours\tabularnewline
\hline 
Andrea Lui (Mat. 850680) & hours\tabularnewline
\hline 
\end{tabular}

\end{itemize}
\pagebreak{}
\end{document}
