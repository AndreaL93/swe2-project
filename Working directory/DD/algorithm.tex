%% LyX 2.1.4 created this file.  For more info, see http://www.lyx.org/.
%% Do not edit unless you really know what you are doing.
\documentclass[english]{article}
\usepackage[T1]{fontenc}
\usepackage[latin9]{inputenc}
\usepackage{babel}
\begin{document}

\subsection{RequestManager}

The RequestManger provides all the method to manage the incoming requests
and them dispatch.
\begin{itemize}
\item \textbf{addRequest(Request)}: insert the Request in the correct position
wrt the starting hour, analize the shared requests into the list and
using GoogleMaps API checks for compatible paths and tries to merge
them.
\item \textbf{modifyRequest(Request, Position destination, Position start,
int passengers)}: if the request was merged then extract the required
request, otherwise delete the request. After that with \textbf{addRequest(Request)}
insert the edited request.
\item \textbf{requestIsReadyNotification()}: every 10 seconds check the
first request in the list if is ready and then notifies to the RideManager
to create a Ride.
\end{itemize}

\subsection{RideManager}

The RideManager manage the pending Rides and Taxi problem report.
\begin{itemize}
\item \textbf{onReadyRequest(Request)}: asks to the QueueManager a available
TaxiDriver with a proper Car, if there is an available Taxi create
a Ride with \textbf{createRide(Request, TaxiDriver)}, otherwise and
the Request is a merged Request using a divide et impera strategy
try to create as few as possible number of Rides to satisfies all
the requests.
\item \textbf{onProblem(TaxiDriver)}: firstly gets the Ride associated to
the TaxiDriver, extract the Request and then use the \textbf{onReadyRequest(Request)}.
\end{itemize}

\subsection{QueueManager}

The QueueManager manage the zones of the city managing them like a
Queue.
\begin{itemize}
\item \textbf{getQueue(int ID)}: using a for clicle get the Queue with the
desired ID.
\item \textbf{getQueue(Position)}: using the MapUtils gets the id of the
Queue and return it with \textbf{getQueue(int ID)}
\item \textbf{getTaxiDriver(Position, int seats)}: gets the Queue using
\textbf{getQueue(Position)}, if the queue has a suitable TaxiDriver
(with a proper car) then return the TaxiDriver, otherwise using the
MapUtils gets the neighbor Queues and retry to get a suitable TaxiDriver
for each Queue, if no TaxiDriver is found return an Exception.
\item \textbf{requestToTaxi(TaxiDriver)}: send a request to the TaxiDriver
and waits 30sec for a reply, if the reply is negative then remove
the TaxiDriver from the queue and push it back to the list and return
false, otherwise return true.
\item \textbf{addNewTaxiDriver(TaxiDriver, Position)}: gets the Queue using
\textbf{getQueue(Position) }and using \textbf{push(TaxiDriver)} insert
back to the list the TaxiDriver.\end{itemize}

\end{document}
