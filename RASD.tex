\documentclass[english]{article}
\usepackage[T1]{fontenc}
\usepackage[latin9]{inputenc}
\usepackage{babel}
\begin{document}

\title{Requirements Analysis and Specification Document\pagebreak{}}

\author{
	Grattarola Daniele, Inajjar Ilyas, Lui Andrea
}

\maketitle
\tableofcontents{}

\pagebreak{}


\section{Introduction}


\subsection{Purpose}

The presented document is the Requirements Analysis and Specification
document (RASD) for the MyTaxiService platform project. 

The main purpose of this document is to analyze the problems that
the new system is to solve and the customer's necessities, to identify
use cases and actors, to describe the functional and non-functional
requirements of the platform along with existing constraints, and
to propose an in-depth specification of the system before moving on
to the more concrete design phase. 

This document is intended for stakeholders, software engineers, and
programmers and is to be used as reference throughout the whole development
of the system. Each phase of the project must be brought on with this
document in mind, and any implementation of the system must be clearly
designed to reflect the specifications presented here.

The secondary audience for this document includes system maintainers
and developers who wish to integrate the platform's services within
their own software.


\subsection{Scope}

The aim of this project is to create the MyTaxiService platform, a
web based information system to manage the taxi service of Pallet
Town. 

The purpose of the platform is:
\begin{enumerate}
\item To provide users with a web application and a mobile application,
with which to easily make use of the town's taxi service.
\item To provide taxi drivers with a mobile application, with which to manage
customers' requests (submitted through the above mentioned systems)
\end{enumerate}
In short, the platform is to be an integrated infrastructure to manage
the initial interaction between customers and taxi drivers, to aid
the former in requesting a taxi ride, and to enable the latter in
managing such requests. 

The platform will therefore be used exclusively to support the town's
taxi service, and is aimed at making the taxi request process much
efficient by reducing overall costs, communication overhead and potential
requests overloads. 

The platform will implement all necessary functions on both the users'
side and the drivers', while keeping the two aspects separate and
unaware of each other. It will also be given a particular focus to
the storage of user data, in order to maximize privacy and security,
and to the extensibility of the system, in order to make the platform
flexible to potential changes in the requirements. 


\subsection{Definitions, acronyms, and abbreviations}

Throughout this document, the following definitions will be applied
without further explanations:
\begin{itemize}
\item \textbf{Platform}: the set of software applications and hardware infrastructure
that make up the MyTaxiService service; these include the back-end
server software, the web application and the mobile application used
by the customers, the mobile application used by the taxi drivers,
and all the necessary hardware needed to run the mentioned software
and any needed support software. 
\item \textbf{System}: the software run on the back-end server of the platform
which is used to handle the communication between the user applications.
The term also addresses all the necessary software components that
are needed to store data, perform calculations and manage the hardware
(e.g. an operating system).
\item \textbf{User-side application}: software run on a personal device
which is used to send taxi requests to the system and to handle the
system's replies. It is designed to be used by customers (see below)
and can either be a mobile application (run on a smartphone or tablet)
or a web application (run on any personal device through an Internet
browser). 
\item \textbf{Taxi-side application}: software run on a personal device
which is used to manage taxi requests forwarded by the system and
to reply to the system with information on how to handle the requests.
It is designed to be used by taxi drivers (see below) and is a mobile
application (run on a smartphone or tablet). 
\item \textbf{Taxi driver}: the owner of a taxi license in Pallet Town,
who uses the taxi-side application to interact with the platform.
\item \textbf{Customer}: a person which intends to exploit the taxi service
of the town, and who uses the user-side applications to interact with
the platform.
\end{itemize}
The following acronyms will also be used in place of the extended
form:
\begin{itemize}
\item \textbf{RASD}: Requirement Analysis and Specification Document
\end{itemize}
Finally, in order to increase readability the following abbreviations
will be used in place of the full words or sentences:
\begin{itemize}
\item \textbf{e.g.: }\textit{exempli gratia}
\end{itemize}

\subsection{References}

This document contains references to other documents which are part
of the complete set of deliverables used to describe and design the
project. The references are intended as follows: 
\begin{itemize}
\item \textbf{Document 1}: ..
\item \textbf{Document 2}: ...
\end{itemize}

\subsection{Overview}

The presented RASD is divided in sections and structured as follows: 
\begin{itemize}
\item \textbf{Section 1 - Introduction}: contains a high level description
of the project, along with its goals, and contains information about
the document itself, such as abbreviations used and the document structure. 
\item \textbf{Section 2 - Overall description}: 
\item \textbf{Section 3 - Requirements}:\end{itemize}

\section{Overall Description}

\subsection{Product Prespective}

The system requirements mainly depends on the web technology necessary to communicate between actors 
(taxi, customer, guest) and the GPS due to locate the taxi customer in the specific sector and also to create the taxi queue. 
The central node system need a server (connected on the web) to elaborate the requests and store the users and taxis information. 
The end-user system relies on the guest operating system (usually a mobile OS like Android, iOS, Windows Phone, blackberry OS, \ldots) 
on the internal GPS system and Internet connection. Of course, all the overall system should only work inside the city.

\subsubsection{System Interfaces}

\begin{itemize}
	\item Google Maps API for taxi ruote
\end{itemize}

\subsubsection{User Interface}

%mockups

\subsubsection{Hardware Interface}

\begin{itemize}
	\item Ethernet connection
	\item GPS system
	\item High fault tolerance of the central node
	\item Firewall: \emph{server store sensitive data}
	\item Router
\end{itemize}

\subsubsection{Software Interface}

\begin{itemize}
	\item Apache Server for CGI
	\item DBMS
\end{itemize}

\subsubsection{Connection Interface}

\begin{itemize}
	\item EDGE, HSDPA, 3G, 4G
	\item HTML, SOAP/REST protocols
\end{itemize}

\subsubsection{Memory Constrains}

\begin{itemize}
	\item Server: 128GB vRAM  requiered, 512GB recommended
	\item 2TB storage capacity + 64GB flash (32GB x2 mirroring) for applications system and server OS
\end{itemize}

\subsubsection{Operations}

\begin{itemize}
	\item Taxi Operations: login, signup, availability notifications, request reply, report
	\item User Operations: login, signup, requests operations, user profile operations (show, edit)
	\item System Operations: notifications (to user and taxi driver), queue managment
	\item Backup operations (scheduled at 4AM every day)
	\item Restore operaions from secondary storage
\end{itemize}

\subsubsection{Site adaptation requirements}

\begin{itemize}
	\item UPS system
	\item 250V/110V Electrical system
	\item Ethernet connection
	\item Anti-thief system, safety room (safety grid on each door and eventually windows) and alarm
	\item Anti-fire system and alaram
	\item Anti-flooding system
	\item Air conditioning
\end{itemize}

\subsection{Product Function}

\subsubsection{Taxi Function}
\begin{itemize}
	\item Taxi Login: the taxi driver logs into the system;
	\item Taxi Signup: the taxi driver provides his personal information, Username and Password, taxi  and driving license to the Operator;
	\item Notification of Availability: with this operation the taxi driver notifies is presence in a specific zone of the town;
	\item Notification of Unavailability: the taxi driver notify the end of his service;
	\item Taking charge: the taxi driver positively reply to a customer request;
	\item Reject request: the taxi driver negatively reply to a customer request;
	\item Report user: the taxi driver report a problem with customer;
\end{itemize}

\subsubsection{User Function}
\begin{itemize}
	\item User Login: the user get logged into the system providing his account information;
	\item User Signup: the user (previously Guest) request a signup to the system providing his personal information;
	\item Request Taxi: the user request a taxi to his location;
	\item Book Taxi: the user reserve a taxi to a specific location at the specific time;
	\item Cancel request: the user cancel the previous request or reservation;
	\item Report taxi: the user report a problem with the taxi driver or the service;
	\item Show Profile: the system provides the user information;
	\item Modify Profile: the user edit his information;
\end{itemize}

\subsubsection{System Function}
\begin{itemize}
	\item Notification to User
	\begin{enumerate}
		\item Incoming taxi: a taxi take charge of the request;
		\item No taxi available;
	\end{enumerate}
	\item Notification to Taxi driver
	\begin{enumerate}
		\item Taxi request: a user need a taxi;
		\item First of the queue: the taxi is the first of the queue of the current zone;
		\item Queue change: notification to the taxi of the new position of the queue;
	\end{enumerate}
	\item Queues management: change the queues after a request reply or new taxi driver in zone;
\end{itemize}

\subsection{User Characteristic}

\subsubsection{User:}
\begin{itemize}
	\item Low knowledge of mobile app use;
	\item Age: over 18;
\end{itemize}

\subsubsection{Taxi Driver:}
\begin{itemize}
	\item Owner of taxi license and driver license;
	\item Low knowledge of mobile app use;
	\item Good knowledge of the local language;
\end{itemize}

\subsection{Constrains}
\begin{itemize}
	\item GPS Coverage;
	\item Privacy policy;
	\item Devices Constrains (presence of GPS and Internet connection);
\end{itemize}

\subsection{Assumption and Dependecies}

\subsection{Apportioning of requirements}

\end{document}